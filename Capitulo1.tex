\chapter{Formulación en Química Inorgánica}

\section{Introducción}\index{Paragraphs of Text}
Desde los primeros albores de las ciencias químicas, los científicos han buscado un método de nomenclatura capaz de representar y reflejar la complejidad y diversidad de la materia que forma el mundo físico que nos rodea. Así, partiendo de los primeros trabajos al respecto del científico inglés \emph{J. Dalton (1766-1844)} los sistemas de nomenclatura han ido evolucionando a la par que la propia química, e igual que cualquier lengua, se han adaptado a los nuevos cambios y descubrimientos, y a la manera en que los propios químicos han desarrollado una ciencia con poco mas de 200 años de historia.\\
En 1919 se crea la \emph{International Union of Pure and Applied Chemistry (IUPAC)} como organismo internacional para, entre otros cometidos, establecer estándares globales de simbología y protocolos operacionales en química.Por tanto, hoy en día la IUPAC es la máxima autoridad en materia de nomenclatura química, y es la encargada de revisar e introducir los cambios pertinentes en los distintos sistemas según el desarrollo natural de la propia ciencia.\\

La última revisión de los protocolos de nomenclatura y formulación tuvo lugar en el año 2005, y fueron publicados, para el caso de la formulación inorgánica, en el “Libro Rojo de la IUPAC” de 2007. Este capítulo contiene todas las nuevas modalidades de nomenclatura y formulación. que incluyen cambios radicales en la manera de nombrar oxoácidos, oxisales y oxoaniones. Como todos los cambios de especial envergadura en ciencia requieren de un periodo de adaptación, se han incluido en los anexos la nomenclatura de dichas especies mediante los sistemas anteriores a las recomendaciones de 2005, de forma que el alumno pueda comprenderlos en el caso de que alguna publicación todavía los contenga.

\section{Sistemas de Nomenclatura}
En el desarrollo de la nomenclatura química han surgido varios sistemas para la construcción de los nombres de los elementos y compuestos químicos. Cada uno de los sistemas tiene su propio conjunto de reglas.
Algunos sistemas son de aplicación general; en cambio, otros han surgido de la necesidad de usar sistemas más especializados en áreas determinadas de la química.
En concreto, en lo referente a la química inorgánica, tres son los sistemas principales de nomenclatura: la nomenclatura de composición, la de sustitución y la de adición.
La nomenclatura de adición es quizás la que puede usarse de forma más generalizada en química inorgánica. La nomenclatura de sustitución puede usarse en determinadas áreas. Estos dos sistemas requieren el conocimiento de la estructura de las especies químicas que van a ser nombradas. En cambio, la nomenclatura de composición puede usarse cuando no es necesario aportar información sobre la estructura de las sustancias, o no se conoce, y sólo se indica la estequiometría o composición.

\subsection{Nomenclatura de Composición}
Esta nomenclatura está basada en la composición no en la estructura. Por ello, puede ser la única forma de nombrar un compuesto si no se dispone de información estructural.\\
El tipo más simple de este tipo de nomenclatura es la llamada estequiométrica. En ella se indica la proporción de los constituyentes a partir de la fórmula empírica o la molecular. La proporción de los elementos o constituyentes puede indicarse de varias formas:\\
\begin{itemize}
	\item Utilizando prefijos multiplicativos (Método Sistemático). 
	\item Utilizando números de oxidación de los elementos (Sistema de Stock, mediante números romanos). 
	\item Utilizando la carga de los iones (mediante los números de Ewens-Basset, números arábigos seguido del signo correspondiente)
\end{itemize}
\subsection{Nomenclatura de Sustitución}
De forma general, en esta nomenclatura se parte del nombre de unos compuestos denominados Hidruros Parentales y se indica, junto con los prefijos de cantidad correspondientes, el nombre de los elementos o grupos que sustituyen a los hidrógenos. Esta nomenclatura es la usada generalmente para nombrar los compuestos orgánicos.
\subsection{Nomenclatura de Adición}
Esta nomenclatura se desarrolló originalmente para nombrar los compuestos de coordinación. Así, se considera que el compuesto consta de un átomo central o átomos centrales con ligandos asociados, cuyo número se indica con los prefijos multiplicativos correspondientes.\\
Los tres sistemas de nomenclatura pueden proporcionar nombres diferentes, pero sin  ambigüedades, para un compuesto dado. La elección entre los tres sistemas depende de la clase de compuesto inorgánico que se trate y el grado de detalle que se desea comunicar.
\section{Tipos de Nomenclatura más utilizados}
A continuación pasaremos a detallar los tipos de nomenclatura mas empleados en Química Inorgánica, independientemente que estos sistemas sean de Composición, Sustitución o Adición\\
\begin{itemize}
	\item \textbf{Nomenclatura de Stock} Se nombra el compuesto seguido de la valencia del elemento central entre paréntesis y en números romanos, si hiciera falta.
	\item[Nomenclatura Sistemática] Se nombra la fórmula del compuesto químico utilizando prefijos para nombrar los subíndices de la fórmula. Dichos prefijos son:\\
	\begin{table}[h!]
		\centering
		\label{tab:nomsist}
		\begin{tabular}{c c|c c}
			Prefijo&Cantidad&Prefijo&Cantidad\\ \hline
			mono-&1&hexa-&6\\ 
			di-&2&hepta-&7\\ 
			tri-&3&octa-&8\\ 
			tetra-&4&nona-&9\\
			penta-&5&deca-&10\\ \hline
		\end{tabular}
		\caption{Prefijos Nomenclatura Sistemática}
	\end{table}
	\item\textbf{Nomenclatura Tradicional} La Nomenclatura Tradicional es un tipo de nomenclatura en desuso, aunque se sigue utilizando masivamente en Oxoácidos, Oxisales y Oxisales Ácidas. Consiste en nombrar el compuesto usando una serie de prefijos y sufijos para indicar la valencia del elemento central. Dichos Prefijos y Sufijos son:
	\begin{table}[h!]
		\centering
		\label{tab:nomtrad}
		\begin{tabular}{c}
			Prefijos y Sufijos\\ \hline
			Hipo- -oso\\
			-oso\\
			-ico\\
			Per- -ico\\ \hline
		\end{tabular}
	\caption{Prefijos y Sufijos de la Nomenclatura Tradicional}
	\end{table}
	\item\textbf{Nomenclatura de Adición} La Nomenclatura de Adición consiste en nombrar los compuestos como adición de iones. Se utiliza sobre todo en Oxoácidos, Oxisales y Oxisales Ácidas
\end{itemize}

\section{Número de Oxidación y Valencia}
	Se denomina valencia a la capacidad combinatoria que tiene un elemento cuando forma compuestos químicos. Normalmente las valencias se comparten entre familias de elementos, aunque  hay excepciones. El número de oxidación es igual que la valencia pero con signo menos cuando el elemento actúa como anión y positivo cuando actúa como catión.
	
\section{Sustancias Binarias}
	Se definen las sustancias binarias como aquellas formadas por dos tipos de elementos. Son sustancias binarias los óxidos, hidruros y sales binarias.

\subsection{Formulación General de Sustancias Binarias}
	Las sustancias binarias se formulan siempre siguiendo las siguientes normas:\\
\begin{enumerate}
	\item Se escriben los símbolos de los elementos que forman el compuesto:\\
\begin{center}
		$PbO$
\end{center}
		\item Se intercambian las valencias de los elementos:\\
\begin{center}
	$Pb_{2}O_4$
\end{center}
		\item Se simplifica si se puede:\\
\begin{center}
	$Pb_{\cancel{2}}O_{\cancel{4}} \rightarrow PbO_2$
\end{center}
\end{enumerate}
\subsection {Óxidos}
Se denominan así a las combinaciones del oxígeno con otro elemento, metálico o no metálico, a excepción de los halógenos.\\

\begin{center}
	$M_{x}O_y$
\end{center}

En estos compuestos, el número de oxidación del oxígeno es -2, mientras que el otro elemento actúa con número de oxidación positivo. Se nombran siguiendo las Nomenclaturas Sistemáticas y de Stock.\\ 

Para hallar la valencia del elemento central se utiliza la siguiente fórmula:

\begin{equation}
	Val(M)= \frac{2\cdot y}{x}
\end{equation}

\begin{table}[h!]
	\centering
	\begin{tabular}{c|cc}
		Fórmula&Nomenclatura Sistemática&Nomenclatura de Stock\\ \hline
		$FeO$&(Mon)óxido de Hierro&Óxido de Hierro (II)\\
		$Fe_{2}O_{3}$&Trióxido de dihierro&Óxido de Hierro (III)\\
		$K_{2}O$&Óxido de dipotasio&Óxido de Potasio\footnote{Puesto que el potasio solo tiene una valencia, ésta se omite }\\
		$P_{2}O_{5}$&Pentaóxido de difósforo&Óxido de Fósforo (V)\\
		$Cu_{2}O$&(Mon)óxido de dicobre&Óxido de Cobre (I)\\ \hline
	\end{tabular}
	\caption{Ejemplos de Óxidos}
\end{table}

\subsection{Peróxidos}
Son combinaciones del anión peróxido $O^{2-}_2$, con un elemento metálico o no metálico. El grupo tiene valencia 2 y \emph{el subíndice del oxígeno no se puede simplificar bajo ningún concepto}. El Compuesto mas simple es el $H_2O_2$, Peróxido de Hidrógeno, que se conoce con el nombre común de Agua Oxigenada. Se nombran siguiendo las nomenclaturas sistemática y de Stock. 
\begin{table}[h!]
	\centering
	\begin{tabular}{c|cc}
		Fórmula&Nomenclatura Sistemática&Nomenclatura de Stock\\ \hline
		$Na_{2}O_{2}$&Dióxido de disodio&Peróxido de Sodio\\ 
		$BaO_{2}$&Dióxido de Bario&Peróxido de Bario\\
		$CuO_{2}$&Dióxido de Cobre&Peróxido de Cobre\\ \hline
	\end{tabular}
	\caption{Ejemplos de Peróxidos}
\end{table}

\subsection{Hidruros Metálicos}
Son combinaciones del Hidrógeno con un metal. La fórmula general es:\\
\begin{center}
	$MH_x$
\end{center}
En los hidruros, \emph{la valencia del elemento central siempre es el subíndice del hidrógeno}, ya que este siempre actúa con valencia 1. Se nombran utilizando las nomenclaturas Sistemática y de Stock.

\begin{table}[h!]
	\centering
	\begin{tabular}{c|cc}
	Fórmula&Nomenclatura Sistemática&Nomenclatura de Stock\\ \hline
	$LiH$&Hidruro de Litio&Hidruro de Litio\\ 
	$CaH_2$&Dihidruro de calcio&Hidruro de Calcio\\
	$FeH_3$&Trihidruro de hierroo&Hidruro de Hierro (III)\\
	$PdH_4$&Tetrahidruro de paladio&Hidruro de Paladio (IV)\\ \hline
	\end{tabular}
	\caption{ejemplos de Hidruros Metálicos}
\end{table}
\subsection{Hidruros No Metálicos}
Son combinaciones del Hidrógeno con no metales de los grupos 13, 14 y 15. Uno de los sistemas de nomenclatura recogidos en las recomendaciones de 2005 de la IUPAC, es la denominada sustitutiva, tal y como se ha comentado al principio. Esta forma de nombrar los compuestos está basada en los denominados \emph{hidruros padres o progenitores}. Éstos son hidruros, con un número determinado de átomos de hidrógeno unidos al átomo central, de los elementos de los grupos 13 al 17 de la tabla periódica.El nombre de los hidruros padres o progenitores están recogidos en la tabla siguiente:
\begin{table}[h!]
	\centering
	\begin{tabular}{c|c||c|c||c|c}\hline
		Nombre&Fórmula&Nombre&Fórmula&Nombre&Fórmula\\ \hline
		$BH_3$&Borano&$CH_4$&Metano&$NH_3$&Azano\\
		$AlH_3$&Alumano&$CH_4$&Silano&$NH_3$&Fosfano\\
		$GaH_3$&Galano&$GeH_4$&Germano&$AsH_3$&Arsano\\
		$InH_3$&Indagano&$SnH_4$&Estannano&$SbH_3$&Estibano\\
		$TlH_3$&Talano&$PbH_4$&Plumbano&$BiH_3$&Bismutano\\ \hline
	\end{tabular}
	\caption{Hidruros Parentales o Progenitores}
\end{table}

\subsection{Sales Binarias Metal-No Metal}
Son combinaciones binarias entre un metal y un no metal con la siguiente fórmula general:\\
\begin{center}
$M_{x}B_y$\footnote{También se consideran sales binarias las combinaciones del anión cianuro ($CN^{-}$) y del catión amonio ($NH^{+}_4$)}
\end{center}
El no metal actúa siempre con la \emph{valencia correspondiente a su estado de oxidación negativo}.Se nombran añadiendo al no metal la terminación \emph{-uro}. La valencia del elemento central se halla teniendo en cuenta la valencia con la que actúa el no metal, y viendo si esta está presente como subíndice en el metal. Se nombran mediante las nomenclaturas Sistemática y Stock.
\begin{table}[h!]
	\centering
	\begin{tabular}{c|cc}
		Fórmula&Nomenclatura Sistemática&Nomenclatura de Stock\\ \hline
		$NaBr$&Bromuro de sodio&Bromuro de Sodio\\ 
		$FeCl_{2}$&Dicloruro de hierro&Cloruro de hierro (II)\\
		$PtI_{4}$&tetrayoduro de platino&Yoduro de Platino (IV)\\
		$Ag_{2}S$&Sulfuro de diplata&Sulfuro de Plata\\ \hline
	\end{tabular}
	\caption{Ejemplos de Sales Binarias Metal-No Metal}
\end{table}
\subsection{Sales Binarias No Metal-No metal}
Son combinaciones binarias entre dos no metales con la siguiente fórmula general:\\
\begin{center}
	$A_{x}B_y$
\end{center}
El más electronegativo se colocará siempre a la derecha y será el que actúe con la menor de las valencias posibles. Se nombrará con la terminación -uro. Aunque también se pueden nombrar con la Nomenclatura de Stock, se recomienda usar la Nomenclatura Sistemática. 

\begin{table}[h!]
	\centering
	\begin{tabular}{c|cc}
		Fórmula&Nomenclatura Sistemática&Nomenclatura de Stock\\ \hline
		$SF_6$&Hexafluoruro de azufre&Fluoruro de Azufre (VI)\\ 
		$PCl_{3}$&Tricloruro de fósforo&Cloruro de Fósforo (III)\\
		$BN$&Nitruto de boro&Nitruto de Boro\\
		$As_{2}S_{5}$&Pentasulfuro de diarsénico&Sulfuro de Arsénico(V)\\
		$ICl_{7}$&heptacloruro de yodo&Cloruro de Yodo (VII)\\ \hline
	\end{tabular}
	\caption{Ejemplos de Sales Binarias No Metal-No Metal}
\end{table}

\subsection{Hidrácidos}
Son combinaciones del Hidrógeno con los elementos de los grupos 16 y 17. Los hidrácidos son en realidad sales de hidrógeno gaseosas disueltas en agua. Así, tendremos dos maneras de nombrarlas, como sal gaseosa o como hidrácido. Para nombrarlo como éste último, se añade la terminación -hídrico.

\begin{table}[h!]
	\centering
	\begin{tabular}{c|cc}
		Fórmula&Nombre de Sal&Nombre de Hidrácido\\ \hline
		$HF$&Fluoruro de Hidrógeno&Ácido Fluorhídrico\\ 
		$HCl$&Cloruro de Hidrógeno&Ácido Clorhídrico\\ 
		$HBr$&Bromuro de Hidrógeno&Ácido Bromhídrico\\ 
		$HI$&Yoduro de Hidrógeno&Ácido Yodhídrico\\ 
		$H_{2}S$&Sulfuro de Hidrógeno&Ácido Sulfhídrico\\
		$H_{2}Se$&Seleniuro de Hidrógeno&Ácido Selenhídrico\\
		$H_{2}Te$&Telururo de Hidrógeno&Ácido Telurhídrico\\ \hline
	\end{tabular}
	\caption{Hidrácidos nombrados como tales y como sales de hidrógeno}
\end{table}

\section{Sustancias Ternarias (I): Hidróxidos Metálicos y Oxoácidos}
Se definen sustancias ternarias como aquellas que estas formadas por tres tipos distintos de átomos. Las sustancias ternarias constituyen uno de los grupos más importantes de toda la química inorgánica.

\subsection{Hidróxidos Metálicos}
Son combinaciones de un metal con el grupo OH, que en su conjunto tiene valencia 1. Su fórmula general es:
\begin{center}
	$M(OH)_x$
\end{center}
Se formula de la misma manera que los compuestos binarios, teniendo en cuenta que el grupo OH se considera como uno y que, por tanto, si hay que añadirle un subíndice éste va en paréntesis. Se nombran mediante las nomenclaturas Sistemática y de Stock:
\begin{table}[h!]
	\centering
	\begin{tabular}{c|cc}
		Fórmula&Nomenclatura Sistemática&Nomenclatura de Stock\\ \hline
		$NaOH$&Hidróxido de sodio&Hidróxido de Sodio\\ 
		$Ca(OH)_{2}$&Dihidróxido de calcio&Hidróxido de calcio\\
		$Fe(OH)_{3}$&Trihidróxido de Hierro&Hidróxido de Hierro (III)\\
		$CuOH$&Hidróxido de cobre&Hidróxido de Cobre (I)\\
		$Mg(OH)_{2}$&Dihidróxido de magnesio&Hidróxido de Magnesio\\ \hline
	\end{tabular}
	\caption{Ejemplos de Hidróxidos Metálicos}
\end{table}
La valencia del elemento central siempre será el subíndice que acompaña al grupo OH (al igual que en los hidruros).
\subsection{Oxoácidos}
Se denominan oxoácidos a aquellos ácidos que contienen oxígeno. Su fórmula general es:
\begin{center}
	$H_{a}X_{b}O_{c}$
\end{center}

Los oxoácidos provienen de añadir agua a un óxido de un no metal. Por tanto, para formularlos se seguirán los siguientes pasos:\\
\begin{enumerate}
	\item Se formula el óxido correspondiente. Para nuestro ejemplo formularemos el Óxido de Nitrógeno (V):
	\begin{center}
		$N_{2}O_5$
	\end{center}
	\item Se le añade agua, procurando disponer los átomos según la fórmula general:
	\begin{center}
		$N_{2}O_5+ H_{2}O \rightarrow H_{2}N_{2}O_6$	
	\end{center}
	\item Se simplifica si se puede:
	\begin{center}
		$H_{\cancel{2}}N_{\cancel{2}}O_{\cancel{6}} \rightarrow HNO_3$
	\end{center}
\end{enumerate}
Según las recomendaciones de la IUPAC 2005, se pueden nombrar de tres maneras distintas: mediante la Nomenclatura Tradicional, Nomenclatura de Adición y Nomenclatura de Hidrógeno.\\
\begin{itemize}
	\item \textbf{Nomenclatura Tradicional o Clásica} Para nombrarlos de este modo, es necesario conocer todos los números de oxidación que
	puede presentar el elemento que actúa como átomo central en la formación de oxoácidos.
	Luego, el número de oxidación que presenta en el compuesto concreto que queremos
	nombrar, se indica mediante sufijo y/o prefijos. Con esta nomenclatura se pueden nombrar hasta cuatro oxoácidos diferentes para un elemento actuando como átomo central. Los prefijos y sufijos que se usan son:\\
	\begin{table} [h!]
		\centering
		\begin{tabular}{c|c||c|c|c|c} \hline
			Pref.&Suf&Cuatro&Tres&Dos&uno\\ \hline
			Hipo-&-oso&Más Bajo&Más Bajo&&\\ \hline
			&-oso&Tercero&Intermedio&Más Bajo&\\ \hline
			&-ico&Segundo&Más Alto&Más Alto&Única\\ \hline
			Per-&-ico&Más Alto&&&\\ \hline
		\end{tabular}
		\caption{Prefijos y Sufijos utilizados en la Nomenclatura Tradicional}
	\end{table}
	Para hallar la valencia del elemento central del oxoácido, se utiliza la siguiente fórmula:
	\begin{equation}
		Val(X)=\frac{2\cdot c-a}{b}
	\end{equation}
	\item \textbf{Nomenclatura de Adición} La nueva nomenclatura de adición introducida por la IUPAC pretende dar información estructural del oxoácido a través de su nombre. Es decir, con la nomenclatura de adición nombramos la estructura de Lewis del oxoácido, lo cual nos dará información extra en el caso de compuestos complejos. Para ello, primero agrupamos los átomos de hidrógeno y oxígeno presentes en el ácido en forma de grupos OH y grupos O solitarios.
	\begin{center}
		$H_{2}SO_{4}  \equiv (OH)_{2}O_{2}S$
	\end{center}
	Se nombran los grupos OH con un prefijo de cantidad y la palabra “hidroxido” los O con prefijo y la palabra “oxido” y por último el nombre del átomo central:
	\begin{center}
		\textit{Dihiroxidooxidoazufre}
	\end{center}
	Cuando un ácido presenta dos entidades dinucleares simétricas, pueden nombrarse estas entidades siguiendo la nomenclatura de adición. Para indicar que son dos entidades, se introduce el nombre entre paréntesis y se utiliza el prefijo \emph{bis-}. Delante, separado por un guión, se nombra el elemento que sirve de puente. Este elemento se nombra anteponiéndole la letra griega \emph{$\mu -$} separada por un guión. Generalmente, en estos compuestos, el elemento que actúa como puente es el oxígeno y se nombra como \emph{-oxido-}. Así, para el caso del Ácido Disulfúrico, $H_{2}S_{2}O_{7}$:
	\begin{center}
		$(OH)O_{2}S-O-SO_{2}(OH) \rightarrow \mu -oxido-bis(hidroxidodioxidoazufre)$
	\end{center}
	\item\textbf{Nomenclatura de Hidrógeno} Consiste en nombrar, en primer lugar, los hidrógenos que contiene el ácido mediante la palabra \emph{hidrogeno-}, precedida por el prefijo de cantidad. A continuación, sin dejar espacios y entre paréntesis, se nombra el anión según la nomenclatura de adición; es decir, en general, se nombran los oxígenos que tiene y se acaba con la raíz del nombre del átomo central acabado en \emph{-ato}.
	\begin{center}
		$H_{2}SO_{4}$ \textit{Hidrógeno(tetraóxidosulfato)}
	\end{center}
	
\end{itemize}

\subsection{Oxoácidos de Interés Especial}

\begin{itemize}
	\item \textbf{Prefijos -orto y -meta} Existen algunos ácidos que de forma natural suelen captar mas de una molécula de agua en el proceso de hidratación desde el óxido correspondiente. Cuando esto ocurre, se le antepone el prefijo \emph{-orto}, si captan tres moléculas de agua, o \emph{-meta} si solo captan una (aunque el prefijo meta se obvia salvo en los cosas en los que el ácido más común es el orto). Por ejemplo, el ácido ortosulfúrico:
	\begin{center}
		$SO_3 + 3\cdot H_{2}O \longrightarrow  H_{6}SO_{6}$
	\end{center}
	\begin{table}[h!]
		\centering
		\begin{tabular}{c|c|c|c}
			Fórmula&Nombre&Fórmula&Nombre\\ \hline
			$H_{5}IO_{6}$&Ácido Ortoperyódico&$HIO_{3}$&Ácido Yódico\\
			$H_{3}PO_{4}$&Ácido Forsfórico&$HPO_{3}$&Ácido Metafosfórico\\
			$H_{3}AsO_{3}$&Ácido Arsenioso&$HPAs_{32}$&Ácido Metaarsenioso\\
			$H_{4}SiO_{4}$&Ácido Silícico&$H_{2}SiO_{3}$&Ácido Metasilícico\\
			$H_{6}TeO_{6}$&Ácido Ortotelúrico&$H_{2}TeO_{4}$&Ácido Telúrico\\ \hline
		\end{tabular}
		\caption{Ejemplos de oxoácidos orto y meta}
		\label{tab:orto}
	\end{table}
	Como puede observarse en el Cuadro \ref{tab:orto}, en el caso de P, As y Si, estos elementos tienen tendencia natural a formar ácidos orto, por lo que se omite el prefijo orto y, por contra, se indica el prefijo meta para el caso de una molécula de agua de hidratación.\\
	
	\item \textbf{Oxoácidos con Doble Número de Átomo Central} Son compuestos que provienen de la hidratación de un óxido de un no metal dimerizado. Se nombran utilizando el prefijo di- o -piro (en desuso):
	\begin{center}
		$2\cdot SO_3 + H_{2}O \rightarrow H_{2}S_{2}O_{7}$
	\end{center}
	\item \textbf{Tioácidos y Peroxoácidos} Los tioácidos provienen de sustituir un oxígeno del oxoácido de partida por azufre. Ej:
	\begin{center}
		$H_{2}S_{2}O_{3}$   Ácido Tiosulfúrico \hspace{1cm} $H_{3}SPO_3$ Ácido tiofosfórico
	\end{center}
	Los peroxoácidos son aquellos que tienen un oxígenos mas que el oxoácido de partida. Ej:
	\begin{center}
		$HNO_4$    Ácido Peroxonítrico \hspace{1cm} $H_{2}SO_5$   Ácido Peroxosulfúrico
	\end{center}
	
\end{itemize}

\section{Iones}
Los iones son especies con carga (ya sea un átomo o un grupo de átomos).En la fórmula de los iones monoatómicos, la carga se expresa con un superíndice a la derecha del símbolo del elemento. Su valor se indica con un número seguido del signo correspondiente $Cu^{2+}$ En los iones poliatómicos, la carga, que se indica igualmente con un superíndice a la derecha del último elemento que forma el ion, corresponde a la suma de los números de oxidación que se atribuye a los elementos que lo constituyen. por ejemplo en el $SO_{4}^{2-}$, la garga (2-) pertenece a todo el ion. Cuando el valor de la carga es uno, ya sea positiva o negativa, sólo se indica con el signo en la fórmula.

\subsection{Cationes Monoatómicos}
Hay dos formas de nombrarlos, basadas en el número de carga o en el número de oxidación:\\
\begin{enumerate}
	
	\item \textbf{Uso del número de carga (sistema Ewens–Basset)} Se nombra el elemento y se indica, seguidamente, el número de la carga entre paréntesis.
	
	\item \textbf{Uso del número de oxidación (sistema de Stock)} Se nombra el elemento y se indica, seguidamente, el número de oxidación entre paréntesis.

\end{enumerate}

\begin{table}[h!]
	\centering\begin{tabular}{c|cc}
		Ion&Ewens-Basset&Stock \\ \hline 
		$Fe^{3+}$&Ion Hierro (3+)&Ion Hierro (III)\\
		$Au^{+}$&Ion Oro (1+)&Ion Oro (I)\\
		$B^{3+}$&Ion Boro (3+)&Ion Boro\\
		$Mg^{2+}$&Ion Magnesio (2+)&Ion Magnesio \\ \hline
	\end{tabular}
	\caption{Ejemplos de Cationes Monoatómicos}
\end{table}
\subsection{Cationes Homopoliatómicos}
Se utiliza la nomenclatura estequiométrica, añadiendole el número de carga correspondiente al nombre del elemento con el prefijo de cantidad y la terminación “-uro”.
\begin{table}[h!]
	\centering
	\begin{tabular}{c|cc}
		Fórmula&Nombre&Nombre Común Aceptado\\ \hline
		$O_{2}^{2-}$&Dioxido(2-)&Peróxido\\
		$I_{3}^{-}$&Triyoduro(1-)&\\
		$N_{3}^{-}$&Trinitruro(1-)&Azida\\
		$S_{2}^{2-}$&Disulfuro(2-)&\\ \hline
	\end{tabular}
		\caption{Ejemplos de Cationes Homopoliatómicos}
\end{table}

\subsection{Oxoaniones}
Son los iones que resultan por la pérdida por parte de los oxoácidos de los iones Hidrógeno(1+) ($H^{+}$). Se pueden nombrar utilizando tres nomenclaturas:\\

\begin{itemize}
	\item\textbf{Nomenclatura Tradicional} Se utilizan los prefijos y sufijos propios de la nomenclatura tradicional, pero con el siguiente cambio:
	\begin{table}[h!]
		\centering
		\begin{tabular}{c|c}
			Ácido&Ion\\ \hline
			-ico&-ato\\
			-oso&-ito\\ \hline
		\end{tabular}
	\end{table}

\begin{center}
	$HClO_4$ \hspace{0.3cm}\textit{Ácido Perclórico}\hspace{1cm}$ClO_{4}^{-}$ \hspace{0.3cm} \textit{Ion Perclorato}\\ \vspace{0,3cm}
	$H_2SO_2$ \hspace{0.3cm}\textit{Ácido Hiposulfuroso}\hspace{1cm}$SO_{2}^{2-}$ \hspace{0.3cm} \textit{Ion Hiposulfito}\\
\end{center}
	Como hay oxoácidos con varios hidrógenos, puede ocurrir que el anión derivado se forme por pérdida de algunos, pero no de todos los hidrógenos. En este caso, se antepone el prefijo hidrogeno-, dihidrogeno-, etc..., según el caso, al nombre del anión.	
\begin{center}
	$H_2SO_4$ \hspace{0.3cm}\textit{Ácido Sulfúrico}\hspace{1cm}$HSO_{4}^{-}$ \hspace{0.3cm} \textit{Ion Hidrogenosulfato}\\
\end{center}
	\item\textbf{Nomenclatura Sistemática} Como hay oxoácidos con varios hidrógenos, puede ocurrir que el anión derivado se forme por pérdida de algunos, pero no de todos los hidrógenos. En este caso, se antepone el prefijo hidrogeno-, dihidrogeno-, etc..., según el caso, al nombre del anión.Se nombran los elementos, indicando el número de cada uno con los prefijos de cantidad. Sería como eliminar los hidrógenos de la Nomenclatura de Hidrógeno de los oxoácidos. Finalmente, se indica la carga del anión mediante el número de carga (sistema Ewens–Basset).
\begin{center}
	$SO_{4}^{2-}$ \hspace{0.3cm} \textit{Tetraoxidosulfato(2-)}\hspace{1cm}$Cr_2O_{7}^{2-}$ \hspace{0.3cm} \textit{Heptaoxidodicromato(2-)}\\
\end{center}
Para los aniones que contienen hidrógeno (oxoaniones ácidos) se puede usar esta nomenclatura, indicando la carga del anión al final del nombre entre paréntesis.
\begin{center}
	$HSO_{4}^{-}$ \hspace{0.3cm} \textit{Hidrógeno(tetraoxidosulfato)(1-)}\hspace{1cm}\\ $H_2PO_{4}^{-}$ \hspace{0.3cm} \textit{Dihidrógeno(tetraoxidofosfato)(1-)}\\
\end{center}
	\item\textbf{Nomenclatura de Adición}Para nombrar estos aniones derivados, se siguen las mismas reglas que para los oxoácidos en cuanto a la forma de nombrar los grupos unidos átomo central, pero se añade el sufijo “-ato” al nombre del elemento que actúa como átomo central y, a continuación, el número de carga del anión entre paréntesis.
	\begin{center}
		$HSO_3^{-} \xrightarrow{\hspace{0,6cm}} [(OH)O_2S]^{-}$ \hspace{0,3cm} \textit{Hidroxidodioxidoazufre (2-)}\\
	\end{center}
	En el caso de oxoaniones no ácidos, la nomenclatura de adición y la sistemática coinciden
\end{itemize}

\section{Sustancias Ternarias (II): Oxisales}
Las oxisales provienen de sustituir los hidrógenos de un oxoácido por un metal, intercambiando sus valencias y simplificando si se puede:
\begin{center}
	$H_2SO_4  \xrightarrow{\hspace{0,3cm}Mg\hspace{0,3cm}} Mg_{\cancel{2}}(SO_4)_{\cancel{2}} \hspace{0,2cm}\equiv\hspace{0,2cm} MgSO_4$
\end{center}
Se pueden nombrar utilizando tres tipos de nomenclatura:\\
\begin{itemize}
	\item \textbf{Nomenclatura Tradicional} Se nombra el oxoanión y, tras la palabra “de”, se indica el nombre del catión al que se le incorpora su valencia mediante el sistema de Stock en el caso que haya ambigüedad
	\begin{table}[h!]
		\centering
		\begin{tabular}{c|c}
			Fórmula&Nombre\\ \hline
			$FeClO_3)_2$&Clorato de Hierro (II)\\
			$MgSO_4$&Sulfato de Magnesio\\
			$Au_2(SO_2)_3$&Hiposulfito de Oro (III)\\
			$NaNO_2$&Nitrito de Sodio\\
			$KMnO_4$&Permanganato de Potasio\\
			$K_2Cr_2O_7$&Dicromato de Potasio\\ \hline
		\end{tabular}
			\caption{Oxisales nombradas por la Nomenclatura Tradicional}
	\end{table}
	\item\textbf{Nomenclatura Sistemática} Se nombra en primer lugar el anión de oxoácido (no se indica la carga) y, tras la palabra “de”, se nombra el catión. La proporción de ambos constituyentes se indica mediante los prefijos multiplicativos.\\ 
	Cuando el nombre de un constituyente comienza por un prefijo multiplicativo o para evitar ambigüedades, se usan los prefijos de cantidad alternativos (bis, tris, tetrakis, pentakis, etc...), colocando el nombre correspondiente entre paréntesis (esto es lo habitual con el oxoanión)\\
	\begin{table}[h!]
		\centering
		\begin{tabular}{c|c}
			Fórmula&Nombre \\ \hline
			$FeClO_3)_2$&Bis(trioxidoclorato) de Hierro\\
			$MgSO_4$&Tetraoxidosulfato de Magnesio\\
			$Au_2(SO_2)_3$&Bis(dioxidosulfato) de dioro\\
			$NaNO_2$&dioxidonitrato de Sodio\\
			$KMnO_4$&Tetraoxidomanganato de Potasio\\
			$K_2Cr_2O_7$&Heptaoxidodicromato de dipotasio\\ \hline
		\end{tabular}
		\caption{Oxisales nombradas por la Nomenclatura Sistematica}
	\end{table}
	\item\textbf{Nomenclatura de Adición} Se nombra el anión de acuerdo a la nomenclatura de adición y, tras la palabra “de”, el catión, utilizando el número de carga correspondiente.
	\begin{table}[h!]
		\centering
		\begin{tabular}{c|c}
			Fórmula&Nombre \\ \hline
			$FeClO_3)_2$&Trioxidoclorato (1-) de Hierro (2+)\\
			$MgSO_4$&Tetraoxidosulfato (2-) de Magnesio (2+)\\
			$Au_2(SO_2)_3$&Dioxidosulfato (2-) de Oro (3+)\\
			$NaNO_2$&Dioxidonitrato (1-) de Sodio (1+)\\
			$KMnO_4$&Tetraoxidomanganato (1-)de Potasio (1+)\\
			$K_2Cr_2O_7$&Heptaoxidodicromato (2-) de Potasio (1+)\\ \hline
		\end{tabular}
			\caption{Oxisales nombradas por la Nomenclatura de Adición}
	\end{table}
\end{itemize}
\section{Sustancias Cuaternarias: Oxisales Ácidas}
Como se ha comentado, algunos oxoácidos están compuestos por varios hidrógenos; si éstos pierden algunos de estos pero no todos, se forman aniones que contienen hidrógeno. Estos aniones cuando se combinan con cationes dan especies neutras llamadas sales (oxisales) ácidas.\\
\begin{itemize}
	\item \textbf{Nomenclatura Tradicional} Se nombra el oxoanión ácido y, tras la palabra “de”, se indica el nombre del catión al que se le incorpora su valencia mediante el sistema de Stock en el caso que haya ambigüedad.
	\begin{table}[h!]
		\centering
		\begin{tabular}{c|c}
			Fórmula&Nombre\\ \hline
			$Cu(HSO_4)_2$&Hidrogenosulfato de Cobre (II)\\
			$CaHPO_4$&Hidrogenofosfato de Calcio\\
			$Fe(H_2PO_3)_3$&Dihidrogenofosfito de Hierro (III)\\
			$FeHBO_3$&Hidrogenoborato de Hierro (II)\\ \hline
		\end{tabular}
		\caption{Oxisales Ácidas nombradas por la Nomenclatura Tradicional}
	\end{table}

	\item\textbf{Nomenclatura de Hidrógeno} Se nombra en primer lugar el anión de oxoácido (no se indica la carga) y, tras la palabra “de”, se nombra el catión. La proporción de ambos constituyentes se indica mediante los prefijos multiplicativos. Cuando el nombre de un constituyente comienza por un prefijo multiplicativo o para evitar ambigüedades, se usan los prefijos de cantidad alternativos (bis, tris, tetrakis, pentakis, etc...), esto es lo habitual con el anión derivado del oxoácido. Además, como el nombre del anión lleva ya paréntesis, el nombre se coloca entre corchetes al utilizar los prefijos alternativos de cantidad.\\
	\begin{table}[h!]
		\centering
		\begin{tabular}{c|c}
			Fórmula&Nombre\\ \hline
			$Cu(HSO_4)_2$&Bis[hidrógeno(tetraoxidosulfato)] de Cobre\\
			$CaHPO_4$&Hidrogeno(tetraoxidofosfato) de Calcio\\
			$Fe(H_2PO_3)_3$&Tris[dihidrogeno(trioxidofosfato)] de Hierro \\
			$FeHBO_3$&Hidrogeno(trioxidoborato) de Hierro \\ \hline
		\end{tabular}
			\caption{Oxisales Ácidas nombradas por la Nomenclatura de Hidrógeno}
	\end{table}
	\item\textbf{Nomenclatura de Adición} Se nombra el anión de acuerdo a la nomenclatura de adición y, tras la palabra “de”, el catión, utilizando el número de carga correspondiente.\\
		\begin{table}[h!]
		\centering
		\begin{tabular}{c|c}
			Fórmula&Nombre\\ \hline
			$Cu(HSO_4)_2$&Hidroxidotrioxidosulfato (1-) de Cobre (2+)\\
			$CaHPO_4$&Hidrotrioxidofosfato (1-) de Calcio (2+)\\
			$Fe(H_2PO_3)_3$&Dihidroxidooxidofosfato (1-) de Hierro (3+)\\
			$FeHBO_3$&Hidroxidodioxidoborato (2-) de Hierro (2+)\\ \hline
		\end{tabular}
			\caption{Oxisales Ácidas nombradas por la Nomenclatura de Adición}
	\end{table}
\end{itemize}
\newpage
\section{Problemas}
\begin{problem}
	Nombra por todas las nomenclaturas posibles los siguientes compuestos: \\ $CuO$, $Cu_2O$, $FeO$, $Fe_2O_3$, $CaO$, $CO_2$, $I_2O_5$, $SO_2$, $Cl_2O_7$, $SO_3$, $Na_2O_2$, $H_2O_2$, $Cu_2O_2$, $Li_2O_2$, $CuO_2$ \\
\end{problem}
\begin{problem}
	Formula los siguientes compuestos:\\ \textit{Óxido de mercurio (II), Óxido de litio, Monóxido de manganeso, Óxido de bario, Trióxido de dicloro, Óxido de bromo (III), Peróxido de potasio, Peróxido de bario, Peróxido de cesio}\\
\end{problem}
\begin{problem}
	Nombra los siguientes compuestos por todas las nomenclaturas posibles:\\ $KH$, $NiH_2$, $NaH$, $FeH_2$, $BeH$, $H_2Se$, $HI$, $NH_3$, $SiH_4$, $HCl$, $H_2S$, $PdH_4$, $Cu(OH)_2$, $Pb(OH)_2$, $NaOH$, $Ni(OH)_3$, $Hg(OH)_2$, $CoCl_3$, $Al_2Se_3$, $CaF_2$, $HBr$, $GaI_3$ \\
\end{problem}
\begin{problem}
	Formula los siguientes compuestos:\\ \textit{Hidruro de Rubidio, Hidruro de Escandio, Hidróxido de Hierro (II), Hidróxido de Cobalto (III), Ácido Sulfhídrico, Ácido Yodhídrico, Hidróxido de Potasio, Carburo de Paladio (II), Nitruro de Cobre (II), Sulfuro de Cadmio, Cloruro de Plata, Silicuro de Magnesio}\\
\end{problem}
\begin{problem}
	Nombra por todas las nomenclaturas posibles los siguientes compuestos:\\ $NaH$, $SiH_4$, $SbH_3$, $PH_3$, $H_2Se$, $NiCl_3$, $LiOH$, $Cu(OH)_2$, $HClO$, $CCl_4$, $Li_2O$, $KCl$, $CaSO_4$, $CaCO_3$, $Ba(OH)_2$, $Na_2O$, $HCl$, $H_2SO_3$, $HNO_3$, $N_2O_5$, $SO_3$, $CaO$, $N_2O_3$, $H_3PO_4$, $Na_3PO_4$, $AgNO_34$, $Mg(OH)_2$, $SO_2$, $CO$,$CO_2$, $NaNO_3$, $GaH_3$, $NaCN4$, $(NH_4)HCO_3$\\
\end{problem}
\begin{problem}
	Formula los siguientes compuestos químicos:\\ \textit{Ácido crómico, Ácido Hiposelenioso, Ácido Brómico, Trihidrógeno(tetraoxidofosfato), Dihidrógeno(trioxidotelurato),  Hidroxidooxidonitrógeno, Hidroxidooxidoiodo, bis(trioxidonitrato) de estroncio, dioxidosulfato de disodio, Tetraoxidofosfato(3-) de calcio (2+), trioxidonitrato(1-) de Oro(3+), sulfato de calcio, sulfato de cobre (II), ácido fosfórico, hipoclorito de sodio, nitrito de hierro (III), Ácido disulfúrico, Tiosulfato de Sodio, Bicarbonato de Sodio, Dihidrogenofosfato de Niquel (III), Tris[hidrogeno(tetraoxidofosfato)] de dialuminio}\\
\end{problem}
\begin{problem}
	Nombra las siguientes especies por todas las nomenclaturas posibles:\\ $SO_2^{2-}$, $ClO_4^{-}$, $HCO_3^{-}$, $BO_3^{-}$, $PO_3^{-}$, $Hg_2^{2+}$, $CN^{-}$, $O_2^{2-}$, $Ca^{2+}$, $HTeO_4^{-}$, $MnO_4^{-}$, $NO_2^{-}$\\
\end{problem}
\begin{problem}
Formula las siguientes especies: \textit{Dioxocarbonato(2-), Ion Hipobromito,Ion Telurito, Dihidrógeno(trioxidofosfato) (2-), Ion Dicromato, Ion Perclorato, Tetraoxidomanganato (1-), Hidroxidodioxidoseleniato(1-), Ion Cobre (II),Ion dimercurio(2+), Hidrogento(dioxidocarbonato) (1-)}
\end{problem}