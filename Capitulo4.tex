\chapter{Termodinámica Química}

\section{Introducción}

La \textbf{Termodinámica} es la parte de la Física que se ocupa del estudio de las relaciones que se establecen entre el calor y el resto de las formas de energía. Entre otras cuestiones la termodinámica se ocupa de analizar los efectos que producen los cambios de magnitudes tales como la temperatura, la densidad, la presión, la masa, el volumen, en los sistemas y a un nivel macroscópico. La base sobre la cual se ciernen todos los estudios de la termodinámica es la circulación de la energía y como ésta es capaz de infundir movimiento. Vale destacar que justamente esta cuestión fue la que promovió el desarrollo de esta ciencia, ya que su origen se debió a la necesidad de aumentar la eficiencia de las primeras máquinas de vapor.\\

Los primeros estudios termodinámicos se deben al ingeniero francés \emph{Nicolas Sadi Carnot (1796-1832)}, quien en 1824 publico un libro titulado \emph{Reflexiones sobre la Fuerza Motriz del Fuego}, donde abordaba la eficiencia de las máquinas de vapor que se utilizaban en la época y los máximos rendimientos que se podían alcanzar con una máquina térmica ideal a la cual se llamó \textbf{Maquina de Carnot}.\\

Las conclusiones obtenidas por Carnot y sus sucesores fueron tan generales y simples que la termodinámica se ha establecido como una disciplina general que se ocupa de describir cómo los sistemas responden a los cambios que se producen en su entorno, pudiéndose aplicar a una infinidad de situaciones tanto de la ciencia como de la ingeniería, como pueden ser: motores, reacciones químicas, transiciones de fase, fenómenos de transporte, agujeros negros… entre otras. 

\section{Termoquímica}

Se define \textbf{Termoquímica} como la parte de la Química que estudia las transferencias energéticas en el transcurso de una reacción química. Dicha energía se manifiesta mayoritariamente en forma de calor, por lo que se puede resumir la termoquímica como la parte de la química que estudia el intercambio calorífico que tiene lugar en una reacción química. Como dicho calor tiene que ver con el contenido de energético de los compuestos químicos intervinientes en el proceso, y teniendo en cuenta que todo sistema químico tiende al estado de mínima energía, la termoquímica se encarga también del estudio de la espontaneidad de las reacciones químicas. 

\section{Sistema Termodinámico}

Se define \textbf{Sistema Termodinámico} como una porción del universo físico que se aísla para su estudio. A la porción del universo físico que se encuentra en contacto con nuestro sistema se le denomina \textbf{Entorno}. Los sistemas termodinámicos se clasifican según el grado de aislamiento que presentan con su entorno. Aplicando este criterio pueden darse tres clases de sistemas:\\

\begin{itemize}
	\item \textbf{Sistema aislado:} Es aquel que no intercambia ni materia ni energía con su entorno, es decir se encuentra en equilibrio termodinámico. Un ejemplo de esta clase podría ser un gas encerrado en un recipiente de paredes rígidas lo suficientemente gruesas (paredes adiabáticas) como para considerar que los intercambios de energía calorífica sean despreciables y que tampoco puede intercambiar energía en forma de trabajo.\\
	
	\item \textbf{Sistema Cerrado:} Es aquel que puede intercambiar energía pero no materia con el exterior. Multitud de sistemas se pueden englobar en esta clase. El mismo planeta Tierra puede considerarse un sistema cerrado. Una lata de sardinas también podría estar incluida en esta clasificación.\\
	
	\item \textbf{Sistema Abierto:} En esta clase se incluyen la mayoría de sistemas que pueden observarse en la vida cotidiana. Por ejemplo, un vehículo motorizado es un sistema abierto, ya que intercambia materia con el exterior cuando es cargado, o su conductor se introduce en su interior para conducirlo, o es provisto de combustible al repostarse, o se consideran los gases que emite por su tubo de escape pero, además, intercambia energía con el entorno. Solo hay que comprobar el calor que desprende el motor y sus inmediaciones o el trabajo que puede efectuar acarreando carga.\\
	
\end{itemize}
	
	Puesto que la gran mayoría de reacciones químicas tienen lugar en disolución, es decir, en un vaso de precipitados, los sistemas químicos se consideran en su mayor parte como \emph{sistemas abiertos}.
	
\section{Variables de Estado}

Una vez determinados los límites de un sistema termodinámico objeto de estudio o investigación, se procede a determinar y cuantificar su estado. En Termodinámica, se denominan \emph{Variables de Estado} a aquellas variables que caracterizan un sistema. Dichas variables son variables macroscópicas, perfectamente medibles en un laboratorio por métodos simples y que caracterizan un estado microscópico concreto. Por ejemplo, en un sistema termodinámico formado por un gas, las variables de estado son la \emph{Presión, Volumen y Temperatura}.\\

Las variables de estado tienen una característica fundamental e importantísima en un proceso: \textbf{Solo dependen de los estados inicial y final del sistema, nunca del proceso seguido}. Esto es de una importancia vital a la hora de estudiar un sistema químico, pues durante una reacción química las variables de estado van a depender exclusivamente de los estados inicial y final, esto es, reactivos y productos, simplificando enormemente el tratamiento teórico de los mismos.

